\section{Disscusion}
Here, we will share some of our viewpoints with you. 

\subsection{Disscusion of value}
As a computational model inspired by the biological nervous system, neural networks can simulate and process complex nonlinear relationships, solving many problems that traditional algorithms find difficult to handle. In practical applications, neural networks are widely used in fields such as speech recognition, image processing, and natural language processing. 

In this article, we mainly use neural networks to learn video image text pairs, and generate videos based on text and image prompts. Through this approach, in the future, video creation may no longer require complex editing software and professional teams. With just a text description and a reference image, high-quality video content can be generated. This lowers the threshold for video creation, allowing non professional users to quickly generate high-quality videos that meet their needs. This simplified process not only saves time and costs, but also encourages more people to participate in creative expression, greatly expanding the possibilities of video content production.

Also, in the fields of advertising, education, entertainment, etc., it is of great significance to generate personalized and highly relevant dynamic content by combining user provided text descriptions and reference images. For example, brands can quickly generate advertising videos based on product copy and images; Educational institutions can automatically generate animated teaching videos using textbook content and illustrations; The entertainment industry can use this technology to generate movie trailers or dynamic stories.

In addition, this technology is also expected to drive the development of virtual reality and augmented reality. By generating highly immersive video content, we can provide users with a more realistic interactive experience. In game development, designers can quickly generate dynamic scenes based on scene descriptions and reference images; In virtual tourism, users can generate lifelike attraction display videos by inputting location descriptions and images.

More importantly, text and image driven video generation has significant value in personalization and user experience optimization. For example, in social media, this technology can help users quickly generate videos related to their personal stories or interests, enhancing the attractiveness and interactivity of the content. At the same time, it can also play a role in fields such as healthcare, education, and cultural preservation, such as producing medical animation, reproducing historical events, and digitally displaying cultural relics.

Therefore, it is necessary to study how to use neural networks to generate videos based on text and image prompts.

\subsection{Disscusion of limitation}
Although this technology has great significance, it also has some obvious problems.

Firstly, this technology has a strong dependence on data. Generating high-quality videos requires a large amount of training data, especially in terms of the diversity and accuracy of the generated content. For some specific fields or highly personalized needs, existing datasets may not be sufficient to support models in generating satisfactory results. In addition, training these models requires a large amount of computing resources, which also limits their popularity, especially on devices with weaker computing power.

Secondly, although the generated video can be constructed based on text and image prompts, it still has certain limitations in terms of content coherence, logic, and creativity. Current technology is usually able to generate short clips based on prompts, but generating complex and highly logically structured videos (such as movies or long videos) remains challenging. The development of the plot, the logic of character behavior, and the emotional expression over a long period of time in the video still pose technical challenges. In addition, the technology of generating videos from text and images currently has limited ability to handle specific details. For some highly demanding details, such as high-quality graphic rendering, complex light and shadow effects, or delicate motion capture, existing generative models often cannot achieve the accuracy of traditional video production. This makes it impossible for this technology to completely replace human labor in some high-end creative fields such as film production, professional advertising, etc.

Finally, the controllability and authenticity of video generation are also issues. Although users can provide text and image prompts, the final generated video may deviate significantly from their expectations. The model may not accurately understand certain complex situations or emotional expressions, resulting in generated video effects that do not conform to the user's ideas. Moreover, due to the high degree of automation in content generation, details, backgrounds, or character expressions in videos may appear unnatural or too stiff, affecting the viewing experience. At the same time, the copyright and ethical issues of generating videos also need to be considered. Due to the fact that the generated content is based on existing data and materials, it may involve copyright disputes, especially when unauthorized elements such as images or music are used in the generated work. In addition, automatically generated videos may be abused for the dissemination of false information, parody, or improper purposes, posing certain social and ethical risks.

Overall, although the technology of generating videos based on text and image prompts has enormous potential, there are still challenges in terms of data, technology, controllability, detail processing, and ethics to truly achieve widespread application and high-quality output. The resolution of these issues will directly determine the future development direction and application scenarios of this technology.

\subsection{Disscusion of development}
To address the limitations of video generation technology based on text and image prompts, there are several directions in which the technology can evolve and improve.

To address the strong reliance on data, expanding and diversifying datasets is crucial. Techniques like data augmentation, transfer learning, and few-shot learning can reduce the data required for training, while synthetic data generation can create specialized training sets. Developing lightweight models optimized for efficiency, such as through model pruning or knowledge distillation, can make this technology more accessible to devices with limited processing power.

Improving generative models to produce coherent, structured, and creative long-form content requires advances in model architecture. Integrating multiple models for story generation, character behavior, and emotional expression, along with reinforcement learning and unsupervised learning, could enhance context understanding. Human-in-the-loop feedback mechanisms would help guide content generation in real time. For better detail precision, combining neural networks with traditional rendering techniques can improve graphic rendering, lighting, and motion capture. Generative adversarial networks (GANs) can be further refined for more realistic visuals and smoother transitions.

Enhancing controllability involves refining user input mechanisms, allowing for more detailed prompts and feedback loops. Improving the model’s understanding of context, semantics, and human behavior through advanced natural language processing and computer vision techniques would also increase authenticity. To address ethical and copyright concerns, implementing content moderation and licensing agreements is essential. Transparency and accountability, such as tracking content origins and preventing misuse like deepfakes, will ensure responsible development of this technology.
